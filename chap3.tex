\documentclass[dvipdfmx]{jsarticle}

% 数式
\usepackage{amsmath,amsfonts,amssymb,amsthm}
\usepackage{bm}
\usepackage{physics}
\usepackage{tcolorbox}
\tcbuselibrary{breakable} % 必要に応じて

\newtcolorbox{mybox}[1][]{%
  title=#1,
  fonttitle=\gtfamily\sffamily\bfseries,
  colframe=blue,
  colback=blue!3!,
  breakable, % 長文対応したい場合
}


\begin{document}
\section{なんで分配関数を計算するのか?}

\begin{mybox}[三章前半のゴール:汎化損失$G_n$が$n\to\infty$の極限でどのような振る舞いをするかを知る]
\begin{equation}
    G_n = L(\omega_0) + \frac{1}{n}\qty(\frac{d}{2\beta} + \frac{1}{2}\abs{\xi}^2 - \frac{1}{2\beta}\tr(IJ^{-1})) + \mathrm{o_p}\qty(\frac{1}{n})
\end{equation}    
\end{mybox}


\end{document}