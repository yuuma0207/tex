\documentclass[dvipdfmx]{jsarticle}

% 数式
\usepackage[landscape]{geometry}
\usepackage{amsmath,amsfonts,amssymb,amsthm}
\usepackage{bm}
\usepackage{url}
\usepackage{physics2}
\usepackage{otf}
\usepackage{array}
\usepackage{multicol}
\usepackage{multirow}

% 描画
\usepackage{tikz}
  \usetikzlibrary{intersections,decorations.markings,decorations.pathmorphing,calc,arrows}
  \tikzstyle{mybox} = [draw=black, fill=white, very thick,
    rectangle, rounded corners, inner sep=10pt, inner ysep=10pt]
  \tikzstyle{fancytitle} =[fill=black, text=white, font=\bfseries]

% ハイパーリンク
\usepackage[dvipdfmx]{hyperref}
  \hypersetup{colorlinks=true,linkcolor=black}
\usepackage{pxjahyper}

\advance\topmargin-.8in
\advance\textheight3in
\advance\textwidth3in
\advance\oddsidemargin-1.5in
\advance\evensidemargin-1.5in
\parindent0pt
\parskip2pt

\newcommand{\hr}{\centerline{\rule{3.5in}{1pt}}}
\renewcommand{\arraystretch}{1.1}

\tikzstyle{mybox} = [draw=black, fill=white, very thick,
    rectangle, rounded corners, inner sep=10pt, inner ysep=10pt]
\tikzstyle{fancytitle} =[fill=black, text=white, font=\bfseries]

\begin{document}
\begin{center}
  \huge
  \textbf{
  ベイズ統計の理論と方法(著:渡辺澄夫)  チートシート
  }
\end{center}
\begin{multicols}{3}


% レイアウト参考 https://www.overleaf.com/articles/130-cheat-sheet/ntwtkmpxmgrp



%----------------------------------------1. はじめに----------------------------------------
\begin{tikzpicture}
  \node [mybox] (box){%
    \begin{minipage}{0.31\textwidth}
      \small
      \begin{tabular}{m{0.31\textwidth} m{0.6\textwidth}}
          \textbf{サンプル} & $x^n = (x_1, x_2, \ldots, x_n) \in \mathbb{R}^{N\otimes n}$ \\
          \textbf{確率変数} & $X^n = (X_1, X_2, \ldots, X_n)$ \\
          \textbf{逆温度} & $\beta$ \\
          \textbf{パラメータ} & $w \in W$ \\
          \textbf{ハイパーパラメータ} & $\phi$ \\
          \hline
          \textbf{真の分布} & $q(x)$ \\
          \textbf{推測した分布} & $\hat{p}(x)$ \\
          \hline
          \textbf{確率モデル} & $p(x|w)$\\
          \textbf{事前分布} & $\varphi(w)$ \\ 
          \textbf{事後分布} & $p(w|X^n) := \frac{1}{Z_n(\beta)}\varphi(w)\prod_i p(X_i|w)^\beta$ \\
          \textbf{分配関数} & $Z_n(\beta) := \int \varphi(w)\prod_i p(X_i|w)^\beta dw$ \\
          \textbf{周辺尤度} & $Z_n(1) = p(X^n)$\\
          \textbf{予測分布} & $p^*(x) := \mathbb{E}_w[p(x|w)] = p(x|X^n)$\\
          \hline
          \textbf{サンプルの現れ方に対する平均} & $\mathbb{E}[f(X^n)] := \int f(x^n) \prod_i q(x_i) dx_i$ \\
          \textbf{真の平均} & $\mathbb{E}_X[f(X)] := \int f(x)q(x)dx$ \\
          \textbf{事後分布による平均} & $\mathbb{E}_w[f(w)] := \int f(w)p(w|X^n)dw$ \\
          \hline
          \textbf{自由エネルギー} & $F_n(\beta) := -\beta^{-1}\log Z_n(\beta)$ \\
          \textbf{(真の)エントロピー} & $S := -\int q(x)\log q(x)dx$ \\ 
          \textbf{経験エントロピー} & $S_n := -\frac{1}{n}\sum_i \log q(X_i)$ \\
          \hline
          \textbf{汎化損失} & $G_n := -\int q(x)\log \hat{p}(x)dx$ \\
          \textbf{経験損失} & $T_n := -\frac{1}{n}\sum_i \log \hat{p}(X_i)$ \\ 
          \hline
          \textbf{自由エネルギーと汎化損失の関係} & $\mathbb{E}[G_n] = \mathbb{E}[F_{n+1}(1)] - \mathbb{E}[F_n(1)]$ \\ 
          \hline
          \textbf{指数型分布} & $p(x|w) = v(x)\exp(f(w)\cdot g(x))$ \\
          \textbf{共役な事前分布} & $\varphi(w|\phi) = z(\phi)^{-1}\exp(\phi \cdot f(w))$ \\
          \multirow{2}{*}{\textbf{事後分布}} & $p(w|X^n) = \varphi(w|\hat{\phi})$ \\
          &s.t. $\hat{\phi} = \phi + \beta\sum_i g(X_i)$\\
          \textbf{予測分布} & $p^*(x) = v(x) z(\hat{\phi} + g(x))/z(\hat{\phi})$ \\
          \hline
          \textbf{Bayes推測} & $\hat{p}(x) := p^*(x)$ \\
          \multirow{2}{*}{\textbf{最尤推測}} & $\hat{p}(x) := p(x|w_\text{ML})$ \\ & s.t. $w_\text{ML} := \arg\max_{w} \prod_i p(X_i|w)$ \\
          \multirow{2}{*}{\textbf{MAP推測}} & $\hat{p}(x) := p(x|w_\text{MAP})$ \\ & s.t. $w_\text{MAP} := \arg\max_{w} p(w|X^n)$ \\
          \textbf{平均プラグイン推測} & $\hat{p}(x) := p(x|\mathbb{E}_w[w])$ \\
          \textbf{変分Bayes法} & 平均場近似した事後分布を使う
      \end{tabular}
    \end{minipage}
  };
  \node[fancytitle, right=10pt] at (box.north west) {1. はじめに};
\end{tikzpicture}



\begin{tikzpicture}
  \node [mybox] (box){%
    \begin{minipage}{0.31\textwidth}
      \leftline{\textbf{Landauの記号(関数)}}
      \small
        \begin{tabular}{m{0.21\textwidth} m{0.7\textwidth}}
          \multirow{2}{*}{\textbf{little o}} & $f(x) = o(g(x))\quad (x \to a)$ \\ & $\Leftrightarrow \lim_{x\to a}f(x)/g(x) = 0$ \\
          \multirow{2}{*}{\textbf{Big O}} & $f(x) = O(g(x))\quad (x\to a)$ \\ & $\Leftrightarrow \exists M > 0, \limsup_{x\to a} |f(x)|/|g(x)| \leq M$ \\
          \hline 
        \end{tabular}\\

      \normalsize	
      \leftline{\textbf{Landauの記号(確率変数列)}}
      \small
        \begin{tabular}{m{0.21\textwidth} m{0.7\textwidth}}
          \multirow{2}{*}{\textbf{little o}} & $X_n = o_p(Y_n)\quad (n\to \infty)$ \\ & $\Leftrightarrow \forall \epsilon>0, \lim_{n\to \infty}P(|X_n|/|Y_n| < \epsilon) = 1$ \\
          \multirow{2}{*}{\textbf{Big O}} & $X_n = O_p(Y_n) \quad (n\to \infty)$ \\ & $\Leftrightarrow \exists M > 0, \lim_{n\to \infty}  P(|X_n|/|Y_n| \leq M) = 1$ 
        \end{tabular}
    \end{minipage}
  };
\end{tikzpicture}

%----------------------------------------2. 基礎概念----------------------------------------
\begin{tikzpicture}
  \node [mybox] (box){%
    \begin{minipage}{0.31\textwidth}
      \small
        \begin{tabular}{m{0.31\textwidth} m{0.6\textwidth}}
        \end{tabular}
    \end{minipage}
  };
  \node[fancytitle, right=10pt] at (box.north west) {2. 基礎概念};
\end{tikzpicture}



%----------------------------------------3. 正則理論----------------------------------------
\begin{tikzpicture}
  \node [mybox] (box){%
    \begin{minipage}{0.31\textwidth}
      \small
        \begin{tabular}{m{0.31\textwidth} m{0.6\textwidth}}
        \end{tabular}
    \end{minipage}
  };
  \node[fancytitle, right=10pt] at (box.north west) {3. 正則理論};
\end{tikzpicture}



%----------------------------------------4. 一般理論----------------------------------------
\begin{tikzpicture}
  \node [mybox] (box){%
    \begin{minipage}{0.31\textwidth}
      \small
        \begin{tabular}{m{0.31\textwidth} m{0.6\textwidth}}
        \end{tabular}
    \end{minipage}
  };
  \node[fancytitle, right=10pt] at (box.north west) {4. 一般理論};
\end{tikzpicture}



%----------------------------------------8. 初等確率論の基礎----------------------------------------
\begin{tikzpicture}
  \node [mybox] (box){%
    \begin{minipage}{0.31\textwidth}
      \small
        \begin{tabular}{m{0.31\textwidth} m{0.6\textwidth}}
        \end{tabular}
    \end{minipage}
  };
  \node[fancytitle, right=10pt] at (box.north west) {8. 初等確率論の基礎};
\end{tikzpicture}




\end{multicols}

\end{document}